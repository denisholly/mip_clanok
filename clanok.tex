% Metódy inžinierskej práce

\documentclass[10pt,oneside,slovak,a4paper]{article}

\usepackage[slovak]{babel}
%\usepackage[T1]{fontenc}
\usepackage[IL2]{fontenc} % lepšia sadzba písmena Ľ než v T1
\usepackage[utf8]{inputenc}
\usepackage{graphicx}
\usepackage{url} % príkaz \url na formátovanie URL
\usepackage{hyperref} % odkazy v texte budú aktívne (pri niektorých triedach dokumentov spôsobuje posun textu)

\usepackage{cite}
%\usepackage{times}

\pagestyle{headings}

\title{História a vývoj digitálnych hier\thanks{Semestrálny projekt v predmete Metódy inžinierskej práce, ak. rok 2022/23, vedenie: Ing. Vladimír Mlynarovič, PhD.}} % meno a priezvisko vyučujúceho na cvičeniach

\author{Denis Hollý\\[2pt]
	{\small Slovenská technická univerzita v Bratislave}\\
	{\small Fakulta informatiky a informačných technológií}\\
	{\small \texttt{xhollyd@stuba.sk}}
	}

\date{\small 6. november 2022} % upravte



\begin{document}

\maketitle

\begin{abstract}
\ldots
\end{abstract}



\section{Úvod}

V dnešnej dobe je pojem digitálne hry už celosvetovo známy. Spomínané sú už aj v televíziách, filmoch, seriáloch, správach, ale aj v každodenných konverzáciách so svojimi známymi. Nie je sa čomu čudovať, keďže tvoria celosvetový priemysel v hodnote cez 100 miliárd eur a odhaduje sa, že je už vo svete vyrobených cez 1.2 biliónov hier a približne dve tretiny američanov tieto hry aktívne hráva.

História digitálnych hier je plná experimentovania, inovácie a kreativity a spájame sa s ňou už vyše 50 rokov.~\cite{Sage:2006} V tomto článku samozrejme nemôžeme podrobne popísať každú hru, ale pozrieme sa spolu na tie, ktoré hrali kľúčovú úlohu v postupnom vývoji až po také, ktoré poznáme aj dnes. V sekcii \ref{druha} popíšeme vývoj historicky prvých hier, ako napríklad OXO, alebo Tennis for Two. Ďalej sa v sekcii \ref{tretia} pozrieme na najznámejšie žánre hier a ich význam. Neskôr sa v ďalšej sekcii \ref{stvrta} spolu pozrieme na veľké tituly hier, ktoré dokázali nadchnúť ich hráčov natoľko, že dokázali spraviť aj pokračovania, ktoré sa tiež veľmi dobre uchytili. V predposlednej sekcii \ref{piata} sa pozrieme na najmodernejšie a najkvalitnejšie hry dnešnej doby a porovnáme ich hlavné rozdiely a posun s predošlými generáciami. Budeme sa zaoberať aj spôsobmi hrania moderných hier, ako napríklad VR hry (hry vo virtuálnej realite), ktoré nevidíme len na našich monitoroch, ale reagujú aj na pohyby hlavy, rúk, nôh a podobne. Nakoniec si v závere \ref{zaver} zhrnieme celkový posun a zhodnotíme pozitíva a negatíva vývoju digitálnych hier.

\section{Vývoj prvých digitálnych hier} \label{druha}

Hoci videohry sa dnes nachádzajú v domácnostiach po celom svete, v skutočnosti začali vo výskumných laboratóriách vedcov. Pravdepodobne úplne prvou hrou bola hra OXO, tiež známu ako noughts and crosses alebo tic-tac-toe. Vytvoril ju britský profesor AS Douglas v roku 1952, ako súčasť svojej doktorandskej dizertačnej práce na University of Cambridge.

Po nej nasledovala hra Tennis for Two, ktorú vytvoril William Higinbotham v roku 1958 na veľkom analógovom počítači a pripojenej obrazovke osciloskopu na každoročný deň otvorených dverí pre návštevníkkov v Brookhaven National Laboratory v Uptone, New Yorku. 

Ďalej v roku 1962 v roku 1962 vynašiel Steve Russell spolu so skupinou hackerov na Massachusetts Institute of Technology hru Spacewar!. Ide o vesmírnu bojovú počítačovú hru vyvíjanú pre PDP-1 (Programmed Data Processor-1). Išlo o špičkový počítač, ktorý sa väčšinou vyskytoval na univerzitách. Táto hra bola prvou hrou, ktorú bolo možné hrať na viacerých počítačových inštaláciách. Spacewar! je považovaná za jednu z najdôležitejších hier celého sveta. Jej realistické zrýchlenie a stratégie orbitálnej mechaniky robili Spacewar! mimoriadne populárnu až do sedemdesiatych rokov minulého storočia.~\cite{Kinephanos:2015}

V roku 1967 vývojári zo Sanders Associates, Inc., na čele s Ralphom Baerom, vynašli prototyp multiplayerového multiprogramového videoherného systému, ktorý bolo možné hrať na televízore. Bol známy ako The Brown Box. Jedna z 28 hier Odyssey bola inšpiráciou pre Atari's Pong , prvú arkádovú videohru, ktorú spoločnosť vydala v roku 1972. V roku 1975 vydala spoločnosť Atari domácu verziu hry Pong, ktorá bola rovnako úspešná ako jej arkádový náprotivok. V roku 1977 Atari vydalo Atari 2600 (tiež známy ako Video Computer System), domácu konzolu, ktorá obsahovala joysticky a vymeniteľné herné kazety, ktoré hrali viacfarebné hry, čím efektívne odštartovala druhú generáciu videoherných konzol.

Po počiatočnom návale klonov Pong a vesmírnych strieľačiek už koncom sedemdesiatych rokov hraniu dochádzali nápady. Potom prišiel Pac-Man, ktorý zjednotil všetkých hráčov (mužov aj ženy) okolo jednej arkádovej skrinky. Je zodpovedná za hltanie viac mincí ako ktorákoľvek iná arkádová hra a je jednou z najdlhšie fungujúcich hier všetkých čias.

Videoherný priemysel mal na konci 70. a začiatkom 80. rokov niekoľko významných míľnikov, vrátane:
\begin{itemize}
\item Vydanie arkádovej hry Space Invaders v roku 1978
\item Uvedenie Activision, prvého vývojára hier tretej strany (ktorý vyvíjal softvér bez výroby konzol alebo arkádových skríň), v roku 1979
\item Predstavenie nesmierne populárneho Pac-Mana v Japonsku
\item Nintendo tvorba Donkey Konga, ktoré predstavilo svetu postavu Mária
\item Microsoft vydal svoju prvú hru Flight Simulator
\end{itemize}

\section{Veľké tituly hier} \label{tretia}

\section{Žánre počítačových hier} \label{stvrta}

\section{Moderné hry} \label{piata}

\section{Záver} \label{zaver} % prípadne iný variant názvu

%\acknowledgement{Ak niekomu chcete poďakovať\ldots}

% týmto sa generuje zoznam literatúry z obsahu súboru literatura.bib podľa toho, na čo sa v článku odkazujete
\bibliography{literatura}
\bibliographystyle{plain} % prípadne alpha, abbrv alebo hociktorý iný
\end{document}
